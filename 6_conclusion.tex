\chapter{Conclusion} \label{ch:conclusion}

This thesis set out to examine how architectural choices and, more decisively, loss-objective design shape an autoencoder’s ability to preserve in-between instances (IBIs), points that bridge clusters and embody transitional structure that classic dichotomies of “inlier vs. outlier” tend to miss. The program of work spanned controlled architectural exploration, a comparative study of unsupervised loss functions (including a novel, differentiable soft trustworthiness objective), and supervised extensions, all evaluated on carefully constructed 2D/3D synthetic datasets with explicit IBI ground truth.

Addressing RQ1, the architectural study clarifies that in two dimensions deeper/wider networks progressively reduce MSE and, within a reasonable capacity range, preserve both cluster geometry and the intended bridging role of IBIs. This relationship weakens in three dimensions, where low MSE does not always coincide with high-quality IBI placement or faithful cluster topology, an indication that once a minimum capacity threshold is exceeded, optimization dynamics and objective choice dominate the geometry of the learned manifold. Pragmatically, the study selects 2–32–16–8–1 (2D) and 3–64–32–16–2 (3D) as robust, balanced baselines for subsequent loss analyses. These choices combine sufficient expressive power with acceptable overfitting risk under the tested objectives.

Turning to RQ2, the influence of loss design is unambiguous across datasets. MSE consistently produces the most trustworthy embeddings and reconstructions: clusters remain distinct, the global arrangement of structures is retained as well as the bottleneck allows, and IBIs land in visually and semantically appropriate bridging regions. These patterns hold from simple Gaussian to non-linear manifolds, including the 2D/3DMoons where MSE preserves arc continuity and places IBIs in the intended transitional corridor even when reconstructions are slightly too “thin” due to compression. By contrast, cosine similarity maintains acceptable behavior on easy cases but sacrifices geometric fidelity as structures become intricate. Manifolds simplify and relative orientations skew, with IBIs drifting into positions that reflect the simplified geometry rather than the intended transitional semantics. KLD performs weakest overall: across settings it collapses dimensionality and merges clusters, obscuring both global identities and IBI function. Finally, the soft trustworthiness objective proves highly sensitive to neighborhood and batch parameters and tends to compress onto 1D manifolds, yet it uniquely “gets right” certain difficult topologies (e.g., unrolling the 3D Swiss roll, partially disentangling the torus, and separating encapsulated spheres), and in these moments it positions IBIs as convincing bridges. These observations motivate a hybrid strategy in which MSE provides global shape while soft trustworthiness sharpens local neighborhoods around decision-relevant regions.

RQ3 shows that reconstructions never fully preserved original dimensionality: embeddings consistently contracted, losing some information yet still achieving clear cluster separation and plausible IBI placement in latent space. Triplet margin and cosine embedding behaved similarly overall with Moons datasets proving hardest and 3DTorus/3DSphere revealing strong disentanglement and separability. Triplet margin most often edged out cosine by better preserving inter-cluster relationships and more consistently positioning IBIs as credible bridges. These results motivate hybrid objectives that pair supervision with reconstruction (e.g., MSE) to anchor global manifold shape while enforcing class-aware separability and accurate transitional placement.

Methodologically, the thesis adopted qualitative inspection of low-dimensional latents and high-dimensional reconstruction to judge cluster topology and IBI placement. This choice is justified because absolute loss values are not commensurate across objectives and may decouple from structural and IBI fidelity, especially in 3D, where low MSE does not guarantee accurate transitional geometry. The insistence on 2D/3D latents, while restrictive, deliberately favors interpretability and direct inspection of IBI behavior, which is otherwise difficult to capture in a single scalar.

Beyond empirical findings, the thesis contributes a differentiable variant of the trustworthiness measure (“soft trustworthiness”), enabling gradient-based training toward neighborhood-faithful projections. Although volatile as a stand-alone objective, its targeted successes on topologically demanding datasets make it a valuable building block for composite losses.

There are, however, clear limitations. All datasets are synthetic. While they are carefully designed to exercise specific geometric and transitional patterns, they cannot exhaust the heterogeneity of real data, and their IBIs are by construction, not emergent artifacts of messy domains. Latent dimensionality was kept to two or three to enable visual adjudication. This aids interpretability but constrains capacity and may bias results toward methods that favor projection simplicity. Finally, the study mostly avoids multi-term losses. Yet the very analysis points to hybrids as the most promising path forward.

The resulting agenda for future work is direct. First, validate on real-world datasets with plausible IBI phenomena, ideally with expert-curated IBI labels to benchmark placement. Second, operationalize hybrid objectives that pair MSE (global shape) with soft trustworthiness (local neighborhood) and, in supervised or semi-supervised settings, add metric-learning regularizers. This should be coupled with neighborhood-aware batch sampling to stabilize training. Third, move beyond purely visual evaluation by developing quantitative IBI-fidelity metrics.

In summary, the study demonstrates that careful loss design matters more than marginal architectural tweaks for preserving the nuanced transitional structure encoded by IBIs. MSE remains a surprisingly strong baseline across regimes, cosine similarity is a competent but brittle stand-in, KLD is ill-suited for faithful structural preservation in this context and the proposed soft trustworthiness, though inconsistent alone, unlocks capabilities exactly where reconstruction objectives falter, suggesting that the right composite can reconcile global fidelity with local neighborhood truthfulness. Supervised losses further strengthen class-aware organization and can preserve IBIs even under tight bottlenecks, pointing to a broad principle: preserve the manifold with reconstruction, respect neighborhoods with local structure terms, and sharpen semantics with supervision. Following this prescription offers a practical recipe for embedding spaces that neither erase bridges nor hallucinate boundaries, a prerequisite for models that must reason about the spaces in-between.
